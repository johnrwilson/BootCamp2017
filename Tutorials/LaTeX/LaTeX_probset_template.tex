\documentclass[letterpaper,12pt]{article}
\usepackage{array}
\usepackage{threeparttable}
\usepackage{geometry}
\geometry{letterpaper,tmargin=1in,bmargin=1in,lmargin=1.25in,rmargin=1.25in}
\usepackage{fancyhdr,lastpage}
\pagestyle{fancy}
\lhead{}
\chead{}
\rhead{}
\lfoot{}
\cfoot{}
\rfoot{\footnotesize\textsl{Page \thepage\ of \pageref{LastPage}}}
\renewcommand\headrulewidth{0pt}
\renewcommand\footrulewidth{0pt}
\usepackage[format=hang,font=normalsize,labelfont=bf]{caption}
\usepackage{listings}
\lstset{frame=single,
  language=Python,
  showstringspaces=false,
  columns=flexible,
  basicstyle={\small\ttfamily},
  numbers=none,
  breaklines=true,
  breakatwhitespace=true
  tabsize=3
}
\usepackage{amsmath}
\usepackage{amssymb}
\usepackage{amsthm}
\usepackage{harvard}
\usepackage{setspace}
\usepackage{float,color}
\usepackage[pdftex]{graphicx}
\usepackage{hyperref}
\hypersetup{colorlinks,linkcolor=red,urlcolor=blue}
\theoremstyle{definition}
\newtheorem{theorem}{Theorem}
\newtheorem{acknowledgement}[theorem]{Acknowledgement}
\newtheorem{algorithm}[theorem]{Algorithm}
\newtheorem{axiom}[theorem]{Axiom}
\newtheorem{case}[theorem]{Case}
\newtheorem{claim}[theorem]{Claim}
\newtheorem{conclusion}[theorem]{Conclusion}
\newtheorem{condition}[theorem]{Condition}
\newtheorem{conjecture}[theorem]{Conjecture}
\newtheorem{corollary}[theorem]{Corollary}
\newtheorem{criterion}[theorem]{Criterion}
\newtheorem{definition}[theorem]{Definition}
\newtheorem{derivation}{Derivation} % Number derivations on their own
\newtheorem{example}[theorem]{Example}
\newtheorem{exercise}[theorem]{Exercise}
\newtheorem{lemma}[theorem]{Lemma}
\newtheorem{notation}[theorem]{Notation}
\newtheorem{problem}[theorem]{Problem}
\newtheorem{proposition}{Proposition} % Number propositions on their own
\newtheorem{remark}[theorem]{Remark}
\newtheorem{solution}[theorem]{Solution}
\newtheorem{summary}[theorem]{Summary}
%\numberwithin{equation}{section}
\bibliographystyle{aer}
\newcommand\ve{\varepsilon}
\newcommand\boldline{\arrayrulewidth{1pt}\hline}


\begin{document}

\begin{flushleft}
  \textbf{\large{Problem Set \#[2]}} \\
  OSM Lab, Zachary Boyd \\
  John Wilson
\end{flushleft}

\vspace{5mm}

\noindent\textbf{Problem 3.1}\\
\textbf{i)}
\begin{multline*}
\frac{1}{4} ( \| x+y \|^2 - \| x - y \|^2) = \frac{1}{4}( \langle x+y,x+y \rangle - \langle x-y,x-y \rangle ) = \\ \frac{1}{4}(\langle x,x \rangle + 2 \langle x,y \rangle + \langle y,y \rangle - \langle x,x \rangle + 2 \langle x,y \rangle - \langle y,y \rangle) = \langle x,y \rangle
\end{multline*}
\textbf{ii)}
\begin{multline*}
\frac{1}{2} ( \| x+y \|^2 + \| x-y \|^2) = \frac{1}{2}(\langle x+y,x+y \rangle + \langle x-y,x-y \rangle) =\\ \frac{1}{2}( \langle x,x \rangle + 2\langle x,y \rangle + \langle y,y \rangle + \langle x,x \rangle -2\langle x,y \rangle + \langle y,y \rangle) = \|x\|^2 + \|y\|^2
\end{multline*}

\noindent\textbf{Problem 3.2}\\
Expansion of the norms into inner products and then separation by linearity yields
\begin{multline*}
\frac{1}{4}(\| x+y \|^2 - \| x - y \|^2 + i \|x - i y \|^2 - i \| x + i y \|^2) = \\ 
\frac {1}{4}(\langle x,x \rangle +\langle x,y \rangle + \langle y,x \rangle + \langle y,y \rangle - \langle x,x \rangle  + \langle x,y \rangle + \langle y,x \rangle - \langle y,y \rangle + i \langle x,x \rangle + \ldots \\ \langle x,y \rangle - \langle y,x \rangle + i \langle y,y \rangle - i \langle x,x \rangle + \langle x,y \rangle - \langle y,x \rangle - i \langle y,y \rangle = \langle x,y \rangle
\end{multline*}

\noindent\textbf{Problem 3.3}\\
By definition 3.3.18, we have that given vectors x and y, $cos \theta = \frac{\langle x, y \rangle}{\| x \| \| y \|}$.\\
\textbf{i)} Note $\sqrt{\int_0^1 x^2 dx} = \frac{1}{\sqrt{3}}$, $\sqrt{\int_0^1 x^{10} dx} = \frac{1}{\sqrt{11}}$, and $\int_0^1 x^6 dx = \frac{1}{7}$. So $\theta = \cos^{-1}\left( \frac{\sqrt{33}}{7}\right)$\\
\textbf{ii)}We have $\sqrt{\int_0^1 x^4 dx} = \frac{1}{\sqrt{5}}$, $\sqrt{\int_0^1 x^{8} dx} = \frac{1}{3}$, and $\int_0^1 x^6 dx = \frac{1}{7}$. So $\theta = \cos^{-1}\left( \frac{3\sqrt{5}}{7}\right)$\\


\noindent\textbf{Problem 3.8}\\
\textbf{i)} Given $\langle f, g \rangle = \frac{1}{\pi}\int_{-\pi}^{\pi}f(t)g(t)dt$, we have the following results: $\langle \cos(t), \cos(t) \rangle = 1, \langle \cos(t), \sin(t) \rangle = 0, \langle \cos(t), \cos(2t) \rangle = 0, \langle \cos(t), \sin(2t) \rangle = 0, \langle \sin(t), \sin(t) \rangle = 1,$ $\langle \sin(t), \cos(2t) \rangle = 0, \langle \sin(t), \sin(2t) \rangle = 0, \langle \cos(2t),\cos(2t) \rangle = 1, \langle \cos(2t),\sin(2t) \rangle = 0, \langle \sin(2t),\sin(2t) \rangle = 1$. Thus the set is orthonormal.\\
\textbf{ii)} $\|t\| = \sqrt{\langle t,t \rangle} = \sqrt{ \frac{1}{\pi}\int_{-\pi}^{\pi}t^2dt } = \frac{\sqrt{6} \pi }{3}$\\
\textbf{iii)} $proj_X(\cos (3t)) = \sum_{f \in S} \langle f, \cos (3t) \rangle f = 0$. It results that $\cos (3t)$ is orthogonal to this set.\\
\textbf{iv)} $proj_Xt = \sum_{f \in S} \langle f, t \rangle f = 2 \sin t - \sin (2t)$\\


\noindent\textbf{Problem 3.9}
Let $\textbf{x} = [x_1 \quad x_2]^T, \textbf{y} = [y_1 \quad y_2]^T$ be real valued vectors. Then $\langle R_{\theta}\textbf{x}, R_{\theta}\textbf{y} \rangle = \cos ^ 2 \theta x_1 y_1 - \cos \theta \sin \theta (x_1 y_2 + x_2 y_1 ) + \sin ^ 2 \theta ( x_2 y_2 +x_1 y_1) + \cos \theta \sin \theta (x_1 y_2 + x_2 y_1 ) + \cos ^2 \theta x_2 y_2 = x_1 y_1 + x_2 y_2 = \langle x, y \rangle$, as desired.\\

\noindent\textbf{Problem 3.10}
\textbf{i)} Let Q be an orthonormal matrix. Then $ \langle Qx, Qy \rangle = \langle x, y \rangle \Rightarrow x^HQ^HQy = x^Hy \Rightarrow Q^HQy = y \Rightarrow Q^HQ = I$. By Proposition 3.2.12, we have that Q is invertible, since the field is of finite dimension n. Thus by uniqueness of inverses, we also have $QQ^H = I$, as desired.\\
\textbf{ii)} Since i) holds, we have $\|Q\textbf{x}\| = \sqrt{\langle x^HQ^HQx \rangle} = \sqrt{\langle x^Hx \rangle} = \|x\|$.\\
\textbf{iii)} Note that by i), $Q^{-1} = Q^H$. We have $\langle Q^Hx, Q^Hy \rangle = x^HQQ^Hy = x^Hy = \langle x, y \rangle$, as desired.\\
\textbf{iv)} Observe that the $i^{th}$ column of Q is equal to $Qe_i$, where $e_i$ is the $i^{th}$ standard basis vector. Then $\langle Qe_i, Qe_j \rangle = \langle e_i,e_j \rangle = \delta_{i,j}$, where $\delta_{i,j}$ is the Kronecker delta. This establishes orthonormality of the columns.\\
\textbf{v)} Let Q be orthonormal. Then we know, since $Q^HQ = 1$, that $\mathrm{det} Q^HQ = 1$. So $1=\mathrm{det} Q^HQ=(\mathrm{det}Q)^2$, so $|\mathrm{det}Q| = 1$.The converse is not true. Take the matrix \[ A= 
\begin{bmatrix}
    1   & 1 & 0 \\
    0   & 1& 0 \\
    0   & 0 &1 
\end{bmatrix}
\]
This matrix has determinant 1 but the second column is not a unit vector, so it is not an orthonormal matrix.\\
\textbf{vi)} Let $Q_1,Q_2 \in M_n(\mathbb{F})$ be orthonormal matrices. Then $\langle Q_1Q_2x,Q_1Q_2y \rangle = x^HQ_2^HQ_1^HQ_1Q_2y = x^HQ_2^HQ_2y = x^Hy = \langle x, y \rangle$, showing that the product is orthonormal.\\

\noindent\textbf{Problem 3.11}
Suppose $x_i$ is a linear combination of the elements $x_j$ of your basis set with $j<i$. When this happens, the difference $x_i - p_{i-1}$ will equal zero, and division by the norm of this difference will cause the problem to be undefined. Thus each element of the basis set must be linearly independent.\\

\noindent\textbf{Problem 3.16}
\textbf{i)} We provide a counterexample. Consider the matrices Q and R given in Example 3.3.11. Modify Q such that \[ Q= 
\begin{bmatrix}
    1   & -1 & 1 & -\sqrt{2} \\
    1   & 1& -1 & 0 \\
    1   & 1 &1 &\sqrt{2}\\
    1 & -1 & -1 & 0
\end{bmatrix}
\]
We still have that $A=QR$ and Q is orthonormal, showing that QR decompositions are not unique.\\
\textbf{ii)} TODO\\


\noindent\textbf{Problem 3.17}
$A^HAx = A^Hb \Leftrightarrow \widehat{R}^H\widehat{Q}^H\widehat{Q}\widehat{R}x = \widehat{R}^H\widehat{Q}^Hb \Leftrightarrow \widehat{R}^H\widehat{R}x = \widehat{R}^H\widehat{Q}^Hb \Leftrightarrow \widehat{R}x = \widehat{Q}^Hb$\\


\noindent\textbf{Problem 3.23}
Note $\| x \| = \|x + y -y\| \leq \|x - y\| + \|y \|$ by the triangle inequality, so $\|x\|-\|y\| \leq \|x-y\|$. Similar logic yields $\|y\|-\|x\| \leq \|y-x\| = \|x-y\|$. Since $\left| \|x\|-\|y\| \right|$ is one of these two cases, the result is immediate.\\

\noindent\textbf{Problem 3.24}
Norm properties: 1: $\| \textbf{x} \| \geq 0$, with equality only if $x=0$. 2: $\|a\textbf{x}\| = |a|\|\textbf{x}\|$. 3: $\|\textbf{x}+\textbf{y}\| \leq \|\textbf{x}\|+\|\textbf{y}\|$.\\
\textbf{i)} Note $|\cdot|$ is a norm, so the three properties hold. Then $\|f\|_{L^1} = \int_a^b |f(t)| dt \geq 0$ since the integrand is non-negative. Similarly, $\int_a^b |f(t)| dt = 0 \Leftrightarrow |f(t)| = 0 \Leftrightarrow f = 0$. So property 1 holds. Property 2: $\|af\|_{L^1} = \int_a^b |af(t)| dt = |a|\int_a^b |f(t)| dt = |a|\|f\|_{L^1}$. Property 3: $\|f+g\|_{L^1} = \int_a^b |f(t)+g(t)| dt \leq \int_a^b |f(t)| + |g(t)|dt = \int_a^b |f(t)| dt + \int_a^b |g(t)| dt = \|f\|_{L^1} + \|g\|_{L^1}$\\
\textbf{ii)} 1: $\|f\|_{L^2} = \left( \int_a^b |f(t)|^2 dt \right) ^{\frac{1}{2}} \geq 0$ since the integrand is non-negative. Similarly, $\left( \int_a^b |f(t)|^2 dt \right) ^{\frac{1}{2}} = 0 \Leftrightarrow |f(t)| = 0 \Leftrightarrow f = 0$. So property 1 holds. Property 2: $\|af\|_{L^2} = \left( \int_a^b |af(t)|^2 dt \right) ^{\frac{1}{2}} = \left( \int_a^b |a|^2|f(t)|^2 dt \right) ^{\frac{1}{2}} = \left( |a|^2 \int_a^b |f(t)|^2 dt \right) ^{\frac{1}{2}} = |a|\left( \int_a^b |f(t)|^2 dt \right) ^{\frac{1}{2}} = |a|\|f\|_{L^2}$. Property 3: First observe that by i), both $\|f+g\|_{L^2}$ and $\|f\|_{L^2}+\|g\|_{L^2}$ are positive quantities. Then we can square both sides and show equivalently that $\|f+g\|_{L^2}^2 \leq \|f\|_{L^2}^2 + 2\|f\|_{L^2}\|g\|_{L^2} + \|g\|_{L^2}^2$. See that $\|f+g\|_{L^2}^2 = \int_a^b |f(t)+g(t)|^2 dt \leq \int_a^b |f(t)|^2 + |f(t)||g(t)| + |g(t)|^2 dt = \int_a^b |f(t)|^2 dt + 2 \int_a^b |f(t)||g(t)|dt + \int_a^b |g(t)|^2 dt = \|f\|_{L^2}^2 + 2\|f\|_{L^2}\|g\|_{L^2} + \|g\|_{L^2}^2$. Thus $\|f+g\|_{L^2}^2 \leq \|f\|_{L^2}^2 + 2\|f\|_{L^2}\|g\|_{L^2} + \|g\|_{L^2}^2$ and $\|f+g\|_{L^2} \leq \|f\|_{L^2} + \|g\|_{L^2}$ and 3 holds.\\
\textbf{iii)} 1: $\|f\|_{L^{\infty}}=\sup_{x \in [a,b]}|f(t)| \geq 0$ with $\sup_{x \in [a,b]}|f(t)| = 0$ only when $f=0$, by properties of supremums and norms. 2: $\|cf\|_{L^{\infty}}=\sup_{x \in [a,b]}|cf(t)| = \sup_{x \in [a,b]}|c||f(t)| = |c|\sup_{x \in [a,b]}|f(t)| = |c|\|f\|_{L^{\infty}}$. Property 3: $\|f+g\|_{L^{\infty}}=\sup_{x \in [a,b]}|f(t)+g(t)| \leq \sup_{x \in [a,b]}|f(t)| + \sup_{x \in [a,b]}|g(t)| = \|f\|_{L^{\infty}} + \|g\|_{L^{\infty}}$ by supremum and absolute value properties.
















\end{document}

